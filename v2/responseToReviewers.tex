\documentclass[10pt]{article}

\title{Reply to reviewers concerning submission BR-Org-18-034 "Investigating soundscapes perception through acoustic scenes simulation"}

\begin{document}

\maketitle

As a preamble, we would like to thank the editor and the reviewer for their comments and suggestions. Following these comments, we made several changes to the article, which are summarized here. The next sections list our answers to each of the reviewer’s comments, with references to the revised manuscript (page, column, and paragraph) where appropriate.

\section{Answer to Academic Editor}

\begin{enumerate}

\item \emph{please follow APA (American Psychological Association) style guidelines (w.r.t. in-text citations, etc.).}

$\rightarrow$ The manuscript now follows the APA citation style. The text have been modified accordingly.

\item \emph{ Please avoid the over-use of acronyms as it makes reading of your article unnecessarily difficult (examples: P@5, ni, i, etc.). This sort of lab jargon is not transparent to the wider readership of BRM. }

$\rightarrow$ Those abbreviations are now in full text, except for some figures for the sake of clarity, in which case the acronyms are detailed in the caption.

\item \emph{I would also like to see comparisons between your approach and other state-of-the-art soundscape analyses (e.g. TAPESTREA, see comment from the reviewer below).}

$\rightarrow$ TODO

\item \emph{You might consider providing a link to a github site with examples of your stimuli. It is hard to imagine what these sound scenes sound like without some real examples.}

The resulting waveforms are available for download \url{https://archive.org/details/soundSimulatedUrbanScene}. The link is now in the introduction. The software platform used for the experiment, the parametrization of the software platform for each generated scene, as well as a 2 dimensional projection of the resulting scenes are available \url{http://soundthings.org/research/urbanSoundScape/XP2014}.

$\rightarrow$

\end{enumerate}

\section{Answer to Reviewer 1}

\begin{enumerate}

\item \emph{The utility of this tool beyond the experiments presented here is not apparent because the flexibility of the system was not demonstrated, for example:
- SimScene itself was only used in the first experiment, and other applications of it were not considered. Design choices vary widely between researchers and domains, meaning that the specific instantiations here may not be generally useful.
- It is not clear how this tool would be expanded to consider other acoustic parameters of interest, beyond level. It might be useful to consider potential relationships to other soundscape softwares such as TAPESTREA, and how the exploration of other acoustic parameters could be made intuitive for experiment participants.}

$\rightarrow$

\item \emph{In the last sentence of the abstract, the distinction of 'physical descriptors' vs. 'global descriptors'  does not have enough context here to be informative.}

$\rightarrow$

\item \emph{In section 2.1 (pg. 5), the citations (particularly 27-31) do not seem relevant for justifying the event/texture distinction in the simulator.}

$\rightarrow$

\item \emph{In the last sentence of the abstract, the distinction of 'physical descriptors' vs. 'global descriptors'  does not have enough context here to be informative.}

$\rightarrow$

\item \emph{In section 2.1 (pg. 5), the citations (particularly 27-31) do not seem relevant for justifying the event/texture distinction in the simulator.}

$\rightarrow$

\end{enumerate}


\end{document}
